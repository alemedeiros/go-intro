% presentation.tex
%   by alemedeiros <alexandre.n.medeiros _at_ gmail.com>
%
% Presentation of Golang for parallel programming course.

\documentclass{beamer}
\usepackage[brazilian]{babel}
\usepackage[utf8]{inputenc}
\usepackage[T1]{fontenc}

\usepackage{graphicx}
\usepackage{subfigure}
\usepackage{float}
\usepackage{array}
\usepackage{textcomp}
\usepackage{listings}
\usepackage{color}

% change font
\usepackage{gillius}

% theme settings
\usetheme{Antibes}     % good
%\usetheme{Bergen}      % check
%\usetheme{Hannover}    % pretty nice, find better colours
%\usetheme{Montpellier} % looks nice

%\usecolortheme{dolphin}
%\usecolortheme{lily}
%\usecolortheme{orchid}
%\usecolortheme{rose}
\usecolortheme{whale}

\beamertemplatetransparentcoveredmedium


% title & author info
\title[Golang]{Introdução a Go}
\subtitle{MO644 -- Programação Paralela}
\author[Alexandre Medeiros]{Alexandre~Medeiros\\{\footnotesize\tt alexandre.n.medeiros@gmail.com}}
\institute[Unicamp]
{
    Instituto de Computação\\
    Universidade Estadual de Campinas\\
}
\date[2014s1]{\today}
\subject{MO644}

\begin{document}

% titleframe
\frame{\titlepage}

\section{Introdução}
\begin{frame}
    {O que é Go?}
    \begin{block}{Golang}
        Go, ou Golang, é uma linguagem de programação desenvolvida por Robert
        Griesemer, Rob Pike and Ken Thompson no Google.
    \end{block}

    \vspace{10pt}
    \centering
    \includegraphics[width=.5\textwidth]{bumper.png}
\end{frame}

\begin{frame}
    {Principais ideias}
    \begin{columns}[c]
        % first column
        \column{.4\textwidth}
        \begin{itemize}[<+->]
            \item Compilação rápida
            \item Execução rápida
            \item Simples de se programar
        \end{itemize}

        % second column
        \column{.6\textwidth}
        \centering
        \includegraphics[width=.6\textwidth]{gopherbw.png}
        \note{Go Gopher foi criado por Renée French.}
    \end{columns}
\end{frame}

\subsection{Funcionalidades}
\begin{frame}
    {Funcionalidades interessantes}
    \begin{itemize}
        \item Estaticamente tipada
        \item {\em Garbage collector}
        \item Sintaxe inspirada em C
    \end{itemize}
\end{frame}

\begin{frame}
    {E mais importante!}
    \pause
    Projetada para ser simples de se criar programas paralelos.
\end{frame}

\section{\em Hands-on!}
\begin{frame}
    {Exemplos}
    Vejamos como Go funciona na prática!
\end{frame}

\subsection{\tt channels}
\begin{frame}
    {Canais de comunicação}
\end{frame}

\subsection{\tt goroutines}
\begin{frame}
    {goroutines}
\end{frame}

\subsection{\tt closure}
\begin{frame}
    {Funções anônimas}
\end{frame}

\subsection{\tt buffered channels}
\begin{frame}
    {Comunicação usando {\em buffers}}
\end{frame}

\subsection{\tt select}
\begin{frame}
    {Recebendo de mais de um canal}
\end{frame}

\subsection{\tt range}
\begin{frame}
    {for-each}
\end{frame}

\subsection{{\tt range} {\em over channels}}
\begin{frame}
    {for-each em um canal}
\end{frame}

\section{E ainda tem mais}
\subsection{Ferramentas}
\begin{frame}
    {Conjunto de ferramentas completo}
    \begin{itemize}[<+->]
        \item go build
        \item go run
        \item gofmt
        \item godoc
        \item go get
    \end{itemize}
\end{frame}
\subsection{Links úteis}
\begin{frame}
    {Referências}
    \begin{itemize}
        \item \url{tour.golang.org}
        \item \url{play.golang.org}
        \item \url{gobyexample.com}
    \end{itemize}
\end{frame}

\section{\em That's all folks}
\begin{frame}
    \centering
    \Large
    Dúvidas?

    \vspace{25pt}
    \includegraphics[width=.2\textwidth]{frontpage.png}
\end{frame}

\begin{frame}
    \centering
    \Large
    Fim!

    \vspace{20pt}
    \normalsize
    Contato: {\tt alexandre.n.medeiros@gmail.com}
\end{frame}

\end{document}
