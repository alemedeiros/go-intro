% presentation.tex
%   by alemedeiros <alexandre.n.medeiros _at_ gmail.com>
%
% Presentation of Golang for parallel programming course.

\documentclass{beamer}
\usepackage[brazilian]{babel}
\usepackage[utf8]{inputenc}
\usepackage[T1]{fontenc}

\usepackage{graphicx}
\usepackage{float}
\usepackage{array}
\usepackage{textcomp}
\usepackage{listings}
\usepackage{color}
\usepackage{hyperref}

% hyper reference commands
\newcommand{\link}[1]{\href{http://#1}{\tt #1}}
\newcommand{\email}[1]{\href{mailto:#1}{\tt #1}}

% change font
\usepackage{gillius}

% theme settings
\usetheme{Antibes}
\usecolortheme{whale}
\beamertemplatenavigationsymbolsempty{}
\beamertemplatetransparentcoveredmedium{}

% listings settings
\usepackage{lstlang0}
\lstset{%
    tabsize=4,
    %frame=single,
    basicstyle=\scriptsize\tt,
    keywordstyle=\scriptsize\bf,
    commentstyle=\scriptsize\it,
    stringstyle=\scriptsize\it,
    captionpos=b,
}
\renewcommand\lstlistingname{Programa}

% title & author info
\title[Go para programação paralela]{Go para programação paralela}
\subtitle{SECOMP}
\author[Alexandre~Medeiros]{Alexandre~Medeiros\\{\footnotesize\email{alexandre.n.medeiros@gmail.com}}}
\institute[Unicamp]
{
    Instituto de Computação\\
    Universidade Estadual de Campinas
}
\date[2014s2]{14 de agosto de 2014}
\subject{SECOMP}

\begin{document}

% titleframe
\frame{\titlepage}

\begin{frame}
    \centering
    Uma versão mais completa desses slides e diversos exemplos podem ser
    encontrados na minha página!

    \vspace{30pt}

    \link{alemedeiros.sdf.org}
\end{frame}

\section{Introdução}
\begin{frame}
    {O que é Go?}
    \only<1>{
        \begin{block}{Golang}
            Go, ou Golang, é uma linguagem de programação {\em Open Source} onde
            é fácil fazer programas simples, confiáveis e eficientes.
        \end{block}
    }

    \only<2>{
        \begin{block}{Criadores}
            A linguagem foi desenvolvida por Robert Griesemer, Rob Pike e Ken
            Thompson no Google.
        \end{block}
    }

    \vfill
    \centering
    \includegraphics[width=.5\textwidth]{bumper.png}
\end{frame}

\begin{frame}
    {Principais ideias}
    \begin{columns}[c]
        % first column
        \column{.4\textwidth}
        \begin{itemize}[<+->]
            \item Compilação rápida
            \item Execução rápida
            \item Simples de se programar
        \end{itemize}

        % second column
        \column{.6\textwidth}
        \centering
        \includegraphics[width=.6\textwidth]{gopherbw.png}
        \note{Go Gopher foi criado por Renée French, mulher do Rob Pike.}
    \end{columns}
\end{frame}

\subsection{Funcionalidades}
\begin{frame}
    {Funcionalidades interessantes}
    \begin{itemize}[<+->]
        \item Sintaxe amplamente inspirada em C
        \item {\em Garbage-collected}
        \item Inferência de tipos de variáveis
    \end{itemize}
\end{frame}

\begin{frame}
    {E mais importante!}
    \pause
    Projetada para ser simples de se criar programas concorrentes.
\end{frame}

\section{É fácil de aprender!}
\begin{frame}[fragile]
    {Hello, World!}

    % Code
    \lstinputlisting[language=Go,caption=hello.go]{examples/hello.go}
    \note{A linguagem por padrão trata todas strings como UTF-8!}
\end{frame}

\subsection{Programas concorrentes}
\begin{frame}
    {goroutines}
    Uma {\tt goroutine} é uma {\em lightweight thread}, gerenciada pelo {\em runtime}.
\end{frame}

\begin{frame}
    {Canais de comunicação}

    Os {\tt channels} são a principal maneira de comunicação e sincronização
    entre {\tt goroutines}.
\end{frame}

\begin{frame}[fragile]
    {Canais de comunicação}

    % Code
    \lstinputlisting[language=Go,caption=channels.go,linerange={8-23}]{examples/channels.go}
\end{frame}

\begin{frame}[fragile]
    {Canais de comunicação}

    % Output
    \verb|$ go run channels.go|
    \lstinputlisting{output/channels.out}
\end{frame}

\subsection{Ferramentas}
\begin{frame}
    {Conjunto de ferramentas completo}
    \begin{itemize}[<+->]
        \item go build
        \item go run
        \item gofmt
        \item godoc
        \item go get
    \end{itemize}
\end{frame}

\subsection{Links úteis}
\begin{frame}
    {Referências}
    \begin{itemize}
        \item \link{tour.golang.org}
        \item \link{gobyexample.com}
    \end{itemize}
\end{frame}

\begin{frame}
    {Divirta-se você também!}
    \centering
    \link{play.golang.org}
\end{frame}

\section{\em That's all folks}
\begin{frame}
    \centering
    \Large
    Dúvidas?

    \vfill
    \includegraphics[width=.2\textwidth]{frontpage.png}
\end{frame}

\begin{frame}
    \centering
    \Large
    Fim!

    \vspace{20pt}
    \normalsize
    Contato: \email{alexandre.n.medeiros@gmail.com}
\end{frame}

\end{document}
